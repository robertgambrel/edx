\documentclass[]{article}
\usepackage[letterpaper, top=1in, bottom=1in, left=1in, right=1in]{geometry}
\usepackage{hyperref}
\usepackage{mathtools}
\hypersetup{hidelinks, pdfauthor={Robert J. Gambrel}, pdftitle={Course Project}}

\setlength{\parskip}{1em}




%opening
\title{Course Project}
\author{Robert Gambrel}

\begin{document}

\maketitle


\section{Case Chosen}
\begin{large}

I chose to consider Mayo Clinic's problem in scheduling operating room use more effectively (\href{https://www.informs.org/Impact/O.R.-Analytics-Success-Stories/Increasing-Surgical-Safety}{https://www.informs.org/Impact/O.R.-Analytics-Success-Stories/Increasing-Surgical-Safety}). The case writeup indicates that the clinic had inefficient scheduling - either due to unused capacity or to procedures taking longer than anticipated.

\section{Estimating Operating Room Time of Known Procedures}

The first thing the facility would need to do is determine how much operating room time to allocate to an upcoming procedure. This could depend on many factors the hospital would have access to - patient age, gender, complicating conditions, prior procedures, and health status. It would also depend on the performing physician - some may just tend to work more slowly than others, which could be captured by including a dummy and interaction terms for each doctor in the scheduling pool. The hospital could combine all of these factors from past procedures to generate a model of how long procedures are expected to take. This would yield both a point estimate and the degree of uncertainty around this estimate. Since the outcome variable is continuous, as it's a measure of time duration, a least squares model is appropriate.

\section{Predicting Influx of Unplanned Procedures}

While most procedures are planned, it is possible that emergency procedures will occur and require operating room time. These would take scheduling priority over previously planned, necessary but not urgent, procedures. If a day was fully booked and an emergency procedure arose, some procedure would have to be bumped and rescheduled. These emergencies could be modeled in a variety of ways - discrete events (Poisson) that potentially occur every hour and affect every surgery later in the day. 

Of course, arrivals for spinal procedures might not be random and memoryless; a traffic accident or natural disaster might cause many patients to need care at once. The patients affected by these would not be independent events in the same way that spinal injuries due to random falls or personal activities would be.

We could also model the likelihood that any given day has any emergency patients. We could use the clinic's past history of emergency operations and include outside data about the region. For example, some days might have an influx of people to the region due to festivals, sporting events, or concerts. Holidays associated with increased road traffic and accidents would also be important to consider. This data could be collected from local news sources but would require a human coder.

\section{Determining the Utility Function}

The analyst must determine whether the model should prioritize minimized unused operating room time or provider time. If the goal is to keep the operating room occupied, then there would be a potential for doctors and their patients to be queued if the time slot they scheduled is delayed by another procedure. This idle time carries a lot of costs - doctors and nurses are expensive, and it is risky to keep patients prepped for surgery under anesthesia for longer than necessary. On the other hand, idle operating rooms are an inefficient use of a clinic's capacity and will still require upkeep and maintenance costs. Patients not seen today will have to be scheduled for later days, which is another queue. How to balance these must ultimately be decided by the hospital; by running multiple simulations under different parameters, administrators can determine which outcomes best balance these costs.

\section{Simulating for Maximum efficiency}

All of these inputs have inherent uncertainty built in. Surgery times vary from expectations, complications arise without warning, unscheduled emergency procedures happen, and sometimes mechanical failures happen that leave operating rooms unavailable. By using this uncertainty to run a simulation of daily surgeries coming in and being successfully completed, the hospital can get a sense of how often it anticipates start time delays over an hour, or doctors to have to stay after 6 pm, or surgeries getting completely bumped and rescheduled. By adjusting the daily scheduling rate and staffing levels of clinical support staff, the hospital could determine what mix of parameters yield the most acceptable outcome.

\section{Scheduling for Maximum efficiency using Continuous Optimization}

Suppose the hospital decides to favor minimizing patient daily wait times at the expense of allowing excess operating room capacity. It decides that by fully scheduling 9 rooms out of 10 (leaving the last one for emergency patients or scheduled patients bumped from other rooms) it will achieve an acceptable level of facility usage and patient queuing and rescheduling. The Mayo writeup indicates that doctors were presented a range of appointment windows. As these windows fill up, the model should be re-optimized for subsequent patients. For example, if the first scheduled appointment the system registers is an 8AM-noon procedure on a patient type that very rarely goes over time, then it should be confident in leaving the afternoon slot open for another procedure. If, however, the patient type runs 2 hours over time in 50\% of cases, then the optimal solution would be to schedule afternoon procedures for other rooms if possible and only use the afternoon slot in this room if necessary. 

\section{Revisiting and Recalibrating}

The parameters of the optimization should be frequently revisited and adjusted based on hospital goals. Perhaps their conservative estimates of excess capacity needs were too conservative, and they could schedule 1 or 2 more procedures a day and still expect to meet their requirements. In addition, new technology makes some procedures easier and faster over time, so they should continuously update their model that estimate procedure time. 



\end{large}
\end{document}
